\subsection{L'Art comme création, la recherche de transcendance} 
\xspace

\sideimage{images/vasarely-feny.jpg}{5cm}{Feny}{Victor Vasarely, 1963}
	L’art est un moyen pour l'homme d'évoluer dans la quête de transcendance. Chacun cherche à se dépasser dans un domaine. Le sportif, par exemple, sonde les limites de son corps quand le scientifique et le philosophe cherchent à approfondir un savoir et une compréhension de l'univers. De même, l'homme croyant en Dieu cherche des réponses aux questions que pose son existence et un moyen d'échapper à la mort. Chaque fois il est question de se dépasser, d'évoluer. L'artiste recherche lui aussi plusieurs formes de transcendances, à travers la technique par exemple. Victor Vasarely a passé sa vie à travailler une technique de dessin pour arriver à des rendus vertigineux proche de la perfection au sens géométrique.
\sideimage{images/casoEscaping.jpg}{7cm}{Escaping critcism}{Caso, huile sur toile, 1874}
L'unité entre l'art et le beau a été une évidence pendant longtemps. L'art classique était poussé par la recherche de l'esthétique, la recherche d'une beauté qui rend hommage à Dieu et à la nature. Le réalisme fût au centre de la recherche artistique depuis le Moyen Âge, surtout au XVIIe siècles, jusqu'au XXe siècle. De nombreux trompe-l'oeil sont produits au cours de cette période. Au Moyen Âge c'est donc en grande partie dans la technicité du peintre que se concentre l'activité artistique. L'artiste recherche une forme de perfection.

\xspace
	Pourquoi cette recherche de réalisme a-t-elle cessé de stimuler le milieu artistique? Plusieurs facteurs semblent entrer en jeu. Dans un premier temps l'Homme a cessé de combattre la nature pour survivre pour arriver à la dominer. Les progrès scientifiques ont amoindri la fascination que l'homme éprouvait à l'égard de la nature. En adjoignant le recul de la religion fasse à l'athéisme plus matérialiste, on arrive à une conception différente de la vision antique de la nature. Elle n'est plus la formidable création de Dieu, mais plutôt le terrain de l'homme. C'est “la mort de Dieu” dont parle Nietzsche. Le Réalisme, mouvement artistique du XIXe siècle, se place au centre du tournant qu'a pris la perception de la nature par l'homme, et plus précisément par l'artiste. Alors que le Néoclassicisme se réfère à la pensée antique d'un idéal parfait, mesuré et équilibré,  le réalisme veut montrer ce qu'il perçoit de manière objective.  Cette pensée est liée aux avancées techniques qui ont lieu à cette époque, notamment la Révolution industrielle, mais aussi du recul de la religion. la 
science prend la place des mythes. Appliquant une méthode dérivée de la méthode scientifique, l'artiste représente ce qu'il voit et non plus des scènes importantes ou des sujets mythologiques. Les paysans ou les gens du peuple deviennent des sujets de tableaux. Par la suite recherche du rendu réaliste est mise à l'écart au profit de rendus plus graphiques avec l'arrivée de l'Impressionnisme, puis du Pointillisme et du Fauvisme.

\sideimage{images/vlaminckBateaux.jpg}{7cm}{Les bateaux-lavoirs}{Maurice Vlaminck, 1906 (le Fauvisme)}
	Par ailleurs, la recherche d'une esthétique réaliste s'inscrit dans des courants artistiques, et l'Histoire des arts nous montre que  passé un temps, le processus de création nécessite du renouveau. Cette dynamique qui maintient la créativité entraîne en contrepartie la mise à l'écart de la plupart des principes de représentation qui ont déjà été exploré. Notons que  notre époque est marquée par une constante recherche de la perfection physique, et ce non plus au niveau artistique, mais au niveau social, vestimentaire et commercial. Nous sommes constamment confrontés à notre image car notre reflet est partout, le miroir n'a pas toujours existé, cela implique une certaine banalisation de la recherche esthétique : l'art ne peut plus se contenter de chercher le beau et le réaliste une fois que la technique a banalisé ce type de rendu et que les domaines du média et de la vente s'en sont emparés. Bien qu'aujourd'hui la rupture soit nette entre l'artiste et l'illustrateur, le moyen Âge ne connaissait pas de différence entre deux types de créations visuelles : l'artiste est l'artisan, art et technique étaient alors étroitement liés.

\xspace
\sideimage{images/ambassadeurs.jpg}{7cm}{Les ambassadeurs}{Hans Holbein le Jeune, 1533. ~\\ National Gallery (Londres)}
	Ce dépassement de soi n'est pas uniquement un dépassement technique, l'artiste veut surpasser son geste, sa créativité, mais aussi sa condition : l'artiste est mortel, sa vie est limitée par le temps alors que son \oe{}uvre, elle, a la possibilité de durer bien plus. L'\oe{}uvre d'art représente donc une occasion de laisser une trace intemporelle de soi, même après la mort. Tous les grands peintres se sont plusieurs fois représentés, dans des autoportraits mais aussi dans d'autres types d'\oe{}uvre dont ils ne sont pas l'objet, comme Albreitch Durër dans \emph{a fête aux couronnes de rose}. Un symbole est récurrent dans ces \oe{}uvre : le crâne qui fait directement référence à la mort. On appelle ces tableaux des vanités. Il est important de bien comprendre cet effet majeur de la capacité d’abstraction de l’homme : l’être humain peut concevoir, c'est-à-dire élaborer des objets mentaux et techniques, et cette capacité le met en face d’une force qui le dépasse et l’effraie : l’homme meurt, et il conçoit l’idée de mort. Cette idée est effrayante, à tel point qu’elle est, selon de nombreux philosophes tels que Blaise Pascal, à la base du comportement individuel et surtout social. Cependant, cette même capacité d’abstraction qui place l’homme devant la peur de mourir lui donne par ailleurs le pouvoir de se dépasser en créant. La notion de création est certainement très délicate car elle nous est naturelle et est souvent confondue avec la notion de production, cependant il est essentiel de garder à l’esprit que créer s’accompagne systématiquement d’une forme interaction avec le monde extérieur, qui est à l’échelle humaine un élément intemporel.

