\subsection{La puissance de l'image et du signe}
\xspace

	L'impact des arts plastiques sur l'homme tient en grande partie à l'efficacité de l'image, et plus précisément du signe. Le signe est la figuration, la représentation d'un objet, d'un être ou d'un concept. Par analogie au langage écrit, le signe est en un sens l'alphabet du langage visuel. “Le signe fait balle sur la rétine. À coup sûr cette carcasse fracassée et incendiée d'automobile qu'aux Etats Unis on a parfois eu l'idée de hisser sur un socle de ciment au bord des routes ou les excès de vitesses sont courants, entraîne la pression du pied sur le frein bien plus sûrement qu'un long discours.” explique René Huyghe dans \emph{Dialogue avec le visible}. Le visuel touche directement le sens le plus important dans la perception de l'environnement, on a naturellement tendance à assimiler ce que l'on voit et ce qui est. Cette force de la figuration visuel semble même tendre à remplacer le langage écrit. Il est certes un peu tôt pour que l'historien en fasse un état de fait, mais on peu constater que les mots de la civilisation du livre reculent devant l'image qui prend de l'ampleur. La phrase au XVIIe siècle est longue, a des périodes, c'est l'époque de la dissertation où la pensée vise à s'amplifier par la forme qui l'exprime jusqu'à atteindre souvent une certaine redondance. Aujourd'hui c'est l'image qui est devenu le premier média, le média pour divertir, pour convaincre, pour vendre. L'image et même son successeur : la vidéo qui permet de mettre l'image en mouvement et de l'accompagner de sons, afin de lui donner toujours plus de crédibilité et d'impact sur le spectateur. L'Art n'a d'ailleurs pas manqué cette évolution technique. Des artistes tels que Nam Jun Paik ont exploité le support vidéo dès ses balbutiements. L'\oe{}uvre de Keith Haring, sur laquelle nous reviendrons, constitue un autre exemple de l'utilisation et de la force du signe et du symbole. De même que la numération arabe, bien plus maniable que la numération romaine, a permis aux mathématiques un bond en avant, la figuration visuelle de notions qui auparavant se développaient dans la pensée a allégé la démarche de l'esprit. Descartes rendait l'algèbre visible et l'inscrivait dans l'espace des graphiques. Leur avantage est de “parler aux yeux”. Descartes lui-même soutenait qu'avec leur aide on pouvait construire tous les problèmes, c'est à dire leur donner une forme, les transformer en image. Il découvrait que l'on peut percevoir plus rapidement, plus globalement et plus précisément avec des images qu'avec des idées. Ainsi, les fonctions et leurs courbes représentatives permettent de visualiser d'emblée des connaissances mathématiques et physiques sans se perdre dans l'énoncé du discours mental souvent bien plus compliqué. 

\xspace
	Notons cependant que cette efficacité de l'image sur l'esprit entraîne dans bien des cas un recul de la logique rationnelle au profit d'une "logique visuelle" factice. La remise en question de ce qui nous est présenté graphiquement, ou plus généralement visuellement, n'est pas spontanée et bien souvent négligée. Un graphique présentant une évolution donnée peut pour des raisons physiques admises avoir à se stabiliser à un certain moment, le plus souvent une croissance ne peut aller au delà d'un certain seuil. La montée du graphique suggère une diagonale dont l'oeil attend une continuité régulière. Une fois le seuil atteint  une horizontale prend brusquement la suite de la diagonale et, bien que ce changement soit attendu par l'esprit, pour le regard il introduit une cassure, une anomalie. En se servant de tels effets on peut susciter des sensations: une impression, une inquiétude, une panique pourtant dépourvues de fondement. Ce procédé est, par exemple, utilisé dans plusieurs œuvres présentés dans le cadre de l’exposition sur la figuration narrative qui a lieu actuellement au Grand Palais : dans l’œuvre d’Adami ci après, l’observateur est déstabilisé par des proportions, des entassements inhabituels qui renforcent son inquiétude et la violence qu’exprime l’artiste. 

\xspace
	L'efficacité du visuel et du signe est aussi liée aux courts-circuits qu'ils ont permis dans le processus de raisonnement. En remplaçant des expériences sensibles particulières par des abstractions généralisées on a permis à l'intelligence humaine de dépasser ses limites: en remplaçant des notions de pensées par des signes on arrive, par exemple à concevoir la notion d'infini qui fait buter l'esprit. On ne sait  concevoir l'infini que par une succession d'additions poursuivie sans fin, mais en y plaçant un symbole conventionnel on se dispense d'une équivalence mentale impossible pour l'utiliser empiriquement. Le signe peut donc aller plus loin que la pensée, ou plutôt permettre à cette dernière de se dépasser en faisant usage de sa capacité d'abstraction. Le Fauvisme et plus encore l'Expressionnisme ont découvert le retentissement psychique de l'image. Lignes et couleurs ont un pouvoir d'évocation qui de la sensation amène à l'émotion. Le Surréalisme est allé plus loin : l'image n'agit  pas seulement sur la pensée mais aussi sur le subconscient.
