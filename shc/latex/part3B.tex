\subsection{L'Art comme média}
\xspace

	Nous avons évoqué précédemment l'efficacité de l'image en tant que média. Bien qu'il soit important de ne pas confondre les arts plastiques avec les domaines de la publicité et de l'information, ces univers ne sont pas pour autant tout à fait hermétiques.\sideimage{images/haringUntitled.jpg}{5cm}{Untitled}{Keith Haring, 1983} Prenons comme exemple l'\oe{}uvre de Keith Haring, certainement l'artiste l’un des artistes les plus importants du Pop Art après Andy Warhol.  L'artiste new-yorkais, né en 1958 et mort du sida à l'age de 31 ans, connaît très vite un grand succès. Très influencé par les univers enfantins de dessins animés, le travail de Keith Haring relie en permanence le milieu des arts plastiques au monde de la rue et de la consommation, touchant un public très large diversifié. Il développe une symbolique colorée, liée au monde des médias et se distingue en créant une iconographie unique, aux formes synthétisées. Sa technique se fonde sur un procédé rapide, il passe rarement plus de deux heures sur une \oe{}uvre, ce qui lui permet une production impressionnante en un sens comparable à la production industrielle. Sa volonté de créer un art populaire toujours plus proche du monde de la consommation se traduit par une prédilection pour des supports hors normes accessibles à tous : le métro, les murs de la ville, les réverbères, jusqu'aux produits dérivés qu'il commercialise lui-même. 
L'étonnant Pop Shop Tokyo (1985) illustre sa volonté de mettre l'art à la portée de tous et nous plonge du sol au plafond dans une vraie boutique qui permettait de diffuser son \oe{}uvre directement au grand public. Au delà de sa technique, qui lui donne ce pouvoir de diffusion, Keith Haring est aussi un artiste très engagé, dont l'\oe{}uvre présente une lutte contre la violence et le racisme. L'\oe{}uvre de cet artiste des années 80 montre bien que les arts plastiques possèdent un pouvoir de diffusion. Keith Haring est loin d'être le seul exemple possible, en réalité 

c'est l'ensemble du Pop art que l'on pourrait étudier sans même être exhaustif. L'art a non seulement la possibilité de diffuser un message, aussi une grande efficacité dans la transmission du message, comme nous l'avons vu avec l'impact du signe et comme le montre le succès fulgurant de l'\oe{}uvre de Keith Haring.
L'art contemporain semble suivre une volonté de transmettre, de toucher, voire dans bien des cas de choquer le spectateur. Le processus vise donc à faire réfléchir le spectateur, cela constitue un nouvel élément dans ce que transmettent les Arts plastiques à l'Homme.
Cette utilisation de la peinture comme moyen transmettre n'en est pas pour autant récente. Nombre d'\oe{}uvres du Moyen Âge occidental des objets de cultes visaient à guider les croyants. Parmi celles-ci on citera le Retable du jugement dernier de Rogier van der Weyden. Ce triptyque monumental est exposé à l'hôpital de Beaune, dans un dortoir où sommeillaient, au temps de sa création, les malades et les blessés. Placé à une extrémité de cet espace, derrière l'autel de telle sorte que les patients pouvaient admirer l'\oe{}uvre depuis leurs lits. Sur la surface intérieure du retable est représentée la scène du jugement dernier rappelant aux malades à la fois leur fin mortelle et leur devoir de tourner leur esprit vers Dieu. Le retable rappelait également que le soin spirituel est aussi important que celui du corps.
\xspace
\centeredimage{images/retableJugement.jpg}{10cm}{Retable du jugement dernier(ouvert)}{Rogier Van der Weyden), vers 1447} \xspace
Par ailleurs la peinture est un moyen efficace de conter : tant d'\oe{}uvres nous racontent les grands passages des différentes mythologies en utilisant une foule de détails qui, dans la scène représentée, acquièrent une protée narrative. C'est le cas par exemple de Mars et Vénus surpris par Vulcain, peint par le Tintoret en 1551 (cf. annexe : fiche de lecture). Les Arts plastiques ont donc bien un pouvoir médiatique au sens où ils peuvent, avec l'efficacité que leur confère leur forme visuelle, transmettre un message, raconter, militer, guider le spectateur ou simplement le pousser à réfléchir.

