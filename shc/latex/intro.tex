%-------------------------------INTRO
	 Le processus de création est un élément fondamental dans la construction psychologique et sociale de l'homme. Cette capacité de créer a permit à l'être humain non seulement de dépasser les limites que lui impose son corps, mais aussi de s'inscrire dans un processus d'évolution non plus biologique, mais historique. L'art est un des principaux signes de cette capacité d'abstraction qui confère à l'homme le pouvoir d'inventer et de créer. Nul doute que l'art est propre à l'homme, mais l'homme lui-même est-il propre à l'art ?

De quoi est fait le lien entre l'homme et l'art ? La relation entre l'art, en tant qu'objet et en tant que concept, et L'homme, en tant qu'artiste et en tant que spectateur, fera l'objet de ce dossier. Nous verrons que l'interaction n'est pas unilatérale, qu'elle n’opère pas non plus en un seul temps. En fait nous verrons que cette interaction s'apparente à un dialogue. Attention cependant : l'art est une notion extrêmement vaste, il serait trop ambitieux, voir même peu judicieux de l'étudier dans sa totalité. C'est donc au quatrième art, celui de la peinture et du dessin, aux arts plastiques que se réduira notre étude. Ajoutons même que cette étude, bien que pouvant être généralisée à de nombreuses autres cultures, s'orientera principalement vers l'art occidental, notamment en ce qui concerne les exemples qui seront traités. Ces restrictions sont nécessaires, compte tenu de l'ampleur du thème abordé, afin de rester cohérent et de pouvoir approfondir certains points intéressants.

\xspace
	La première difficulté qui se pose à l'étude de l'art est la richesse de sa dimension historique. Il n'est pas possible de mener une réflexion sur l'art sans étudier son évolution dans le temps, tant cette dernière est importante. Nous aborderons donc en premier lieu un exemple illustrant l'évolution de l'art à travers les époques. Les informations que nous tirerons de cet exemple nous seront utiles pour étudier ensuite l'évolution de l'objet d'art dans le temps et enfin l'évolution du concept artistique.

	Une fois la dimension temporelle mise en place, nous nous concentrerons sur le dialogue entre l'artiste et l'oeuvre d'art. La production artistique est un moyen d'expression et de réflexion. Nous verrons alors que la nature du lien entre créateur et création repose sur une forte volonté de transcendance, propre à l'homme, certes, mais particulièrement présente dans le processus de création.
L’artiste travaille pour un autre, un spectateur. Une fois exposée, son œuvre cesse de lui appartenir pour exister sous le regard de l’autre. Le rapport entre l’œuvre et le spectateur fondera la troisième partie de ce dossier. Nous verrons tout d’abord la puissance de l’image et du signe, puis comment l’œuvre d’art transmet, guide, et constitue un média. Enfin, nous aborderons un des points fondamentaux de l’attachement des hommes aux œuvres d’art : l’émotion qu’elles provoquent.   
	
