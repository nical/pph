\subsection{La recherche de l'émotion}
\xspace

	Cette volonté caractéristique de l'art contemporain de créer un impact fort chez le spectateur semble mettre les plasticiens d'aujourd'hui dans une tout autre catégorie que ceux d'hier. Il est vrai que l'art d'aujourd’hui déroute, et même bien trop souvent dégoûte. On ne compte plus les artistes qui produisent des \oe{}uvres se fondant sur une forme de laideur. 
La Fontaine de Duchamp, par exemple, a fait l'effet d'une bombe détruisant l'idée que l'art était le reflet du beau. Cette \oe{}uvre, bien qu'étant en quelque sorte l'emblème, le stéréotype de l'art contemporain, n'a pas tout de suite été considérée comme objet d'art. L'urinoir constituait initialement une blague pour épater le New York mondain du salon des Arensberg que fréquentait alors l'artiste. L'\oe{}uvre a d'abord été décrétée “non artistique“ par la Society of Independent Artists.

\sideimage{images/duchamp.jpg}{5cm}{Fontaine}{vMarcel Duchamp, 1917}
Ce n'est que bien plus tard, dans le cadre de la première grande rétrospective dadaïste, au Pasadena Museum of Art de Los Angeles, qu'il est demandé à Marcel Duchamp de récréer la Fontaine. Par son succès l’objet obtient un statut d'\oe{}uvre d'art et Duchamp reçoit pas moins de vingt demandes de nouvelles fontaines. Que retient-on de cette anecdote ? Dans un premier temps, que c'est principalement ce très bref passage de l'histoire de l'art qui a marqué un tournant extrêmement important : l'\oe{}uvre d'art prend une complète indépendance vis à vis de la dimension artisanale. Artisanat qui était, rappelons le, le sens  premier du mot “art” au Moyen Âge. La Fontaine fait en effet partie des “ready made” de l'artiste, nom tout à fait explicite désignant ses \oe{}uvres qu'il n'a pas créé lui même. Retenons par ailleurs le détail suivant : l'\oe{}uvre au départ n'en était pas réellement une, mais plutôt une farce de l'artiste. Etonnant, pour un objet qui est aujourd’hui l'emblème de tout un courant artistique. Il semble que cette \oe{}uvre ait su susciter chez le spectateur, chez le critique d'art et même dans tout le milieu artistique, une émotion ou une réflexion à telle enseigne que les artistes de la figuration narrative ont éprouvé le besoin d’ « assassiner » Marcel Duchamp dans une suite de tableaux qui font référence à son urinoir. 

\centeredimage{images/finMarcel.jpg}{10cm}{Vivre et laisser mourir ou la fin tragique de Marcel Duchamp}{Gilles Aillaud, Eduardo Arroyo et Antonio Recalcati, 1965. }

 Qu'est-ce qui fait le succès de cet urinoir renversé ? \emph{La Fontaine} de Duchamp a en fait renversé les dogmes artistiques. La polémique engendrée par cette invention à entraîné bien des réflexions, et c'est sur ce point précis que l'on peut la rattacher au reste des créations plastiques : l'art de notre temps cherche l'émotion et la réflexion. Ainsi, alors qu'on a longtemps cherché à émouvoir par le beau, c'est aujourd'hui une esthétique bien plus conceptuelle qui est visée. L'art contemporain invente peu dans le domaine de la technique, l'artiste cherche maintenant à inventer dans le concept. Il doit se démarquer, il s'agit d'une course à l'invention qui repose non plus sur un talent technique, mais plutôt sur la recherche d'une idée que personne n'a eue auparavant. Ce caractère novateur entraîne alors tantôt réflexion, tantôt émotion, quoi qu'il en soit le spectateur est interpellé. Et cet accent sur l’esthétique est important car c’est lui qui élargit le public de l’art. En effet l’esthétique touche facilement le spectateur quel que soit son niveau d’éducation, puisqu’elle agit directement au niveau sensoriel sans passer par un langage ou un code particulier.

\xspace
	Remontons le temps. Quelques siècles suffisent pour se rendre compte que cette implication du spectateur n'est pas nouvelle. L'art avant Duchamp cherchait le beau afin d'émouvoir l'homme qui le regarde. Il y a toujours eu en art la recherche d'une esthétique. Bien qu'aujourd’hui il s'agit d'une esthétique du concept, l'artiste a d'abord cherché la beauté dans le monde : la beauté dans Dieu, dans la nature et dans l'homme et son action. S'il y a une beauté dans l'objet d'art, cette beauté n'a de sens que dans l'oeil du spectateur. Ceci est important du point de vue de l'effet de l'art sur le spectateur, mais aussi dans l'autre sens : l'homme est parfois sujet de l'\oe{}uvre, mais il est toujours sujet de l'intention du peintre. L'art s'adresse à l'homme et dépend de l'homme. L'art est un phénomène culturel dans le sen où il est lié à la vie de son artiste, mais aussi parce qu’il est lié à celle de ses spectateurs. En effet une \oe{}uvre d'art est avant tout un objet perçu. Nous avons vu par exemple que le musée représentait une caractéristique majeure du concept artistique d'aujourd'hui, et il en est ainsi parce qu'il a un impact direct sur le mode et le contexte de perception. Le spectateur est le censeur, le sujet final en un sens, et cela fait de lui le juge. Ce terme semble convenir au rôle de l'homme face à l'\oe{}uvre d'art et l'anecdote de la Fontaine de Duchamp en témoigne, tout comme les histoires de ces nombreux artistes qui ne furent reconnus qu'après leur mort. Aujourd'hui, comme à toute les époques, l'objet d'art est soumis au jugement de ceux qui l'observent. 

\xspace
	Daniel Arasse, \emph{dans On n’y voit rien : descriptions}, propose une réflexion sur la façon d'observer, d'analyser et de s'ouvrir aux \oe{}uvres des grands peintres en opposant deux approches différentes. La première, très savante, se base sur une lecture iconographique de l’œuvre tandis que la seconde, celle que défend Arasse, est plus sensualiste : selon lui les écrits et autres références ne doivent intervenir qu'après une observation sensible d'un tableau, sans quoi ils risquent de faire obstacle à la compréhension. Ainsi l’\oe{}uvre d'art est plus qu'un langage symbolique et la connaissance de l’iconographie et de l’histoire de l’art sont certes des outils, mais doivent seconder une exploration sensible de l'\oe{}uvre.
