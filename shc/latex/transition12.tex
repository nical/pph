%----------------Transition entre les poarties 1 et 2

	L’art s’inscrit donc dans un processus historique complexe. L’exemple de la couleur bleue, qui fait figure de détail à l’échelle de toutes les facettes de la création artistique, en témoigne. Et dans cette évolution, le contact entre l’homme et l’art ne demeure pas inchangé, loin de là. Le temps séparant une œuvre et un spectateur donné est important car l’art est un phénomène culturel : la vie morale et psychologique du peintre, les sensations qu’il associe à son geste, tout comme le regard et l’interprétation du spectateur sont autant d’aspect primordiaux, puisqu’ils sont au cœur du contact entre l’homme et l’art, qui dépendent du contexte socioculturel  de création et d’exposition. Il devient donc difficile de généraliser une étude sans prendre en compte l’histoire des arts dans son intégralité. Est-il alors possible d’énoncer une définition précise et objective de l’art ?
