\subsection{Le concept artistique évolue dans le temps}
\xspace

	Il ne faut cependant pas s'arrêter à ce constat. Le rapport entre l'homme et l'objet d'art évolue car l'un des deux évolue, certes mais cette transformation ne concerne pas uniquement l'objet au singulier : le concept artistique se meut lui aussi. Ce qu'on entend par concept artistique est le rapport entre l'homme et l'art plutôt qu'entre l'homme et l'objet d'art.

Il semble que l'art reflète la fascination de l'homme. Jusqu'à la Renaissance, l'art est presque exclusivement religieux. L'artiste rend hommage a Dieu en représentant sa création et en représentant des scènes religieuses. Le culte et l'art sont en effet liés par la volonté de dépasser une réalité matérielle souvent pénible. Par la suite la création artistique se détache peu à peu de la tradition religieuse pour s’émerveiller de la nature non plus comme ouvrage divin mais plutôt comme univers. Les retombées de la Révolution industrielle détachent l'art du domaine du sacré pour en faire un document social et matériel. Ce que nous considérons aujourd'hui comme œuvre d'art n'a pas toujours eu ce statut.
\sideimage{images/masquewobe.jpg}{4cm}{}{Masques Wobés africains ayant inspiré Pablo Picasso}
Prenons l’exemple des masques africains devenus objets d’art dans la culture européenne après la « redécouverte » qu’en ont fait les cubistes. Ces derniers en ont alors fait la  promotion au travers de leurs propres collection. C'est en grande partie le contexte d'exposition qui transforme l'objet artistique. Au Moyen Âge un tableau était un objet de culte ou un objet décoratif, mais pas un objet à proprement parler “artistique”. Nous avons vu l'importance du contexte de création, n'oublions pas l'importance du contexte d'exposition. Si dans un premier temps on se pose la question: “pour qui était destinée cette \oe{}uvre? Dans quelles conditions d'exposition?”, on n'oublie pas pour autant que la dimension temporelle de l'objet artistique implique que l'on se demande aussi “Qui observe cette \oe{}uvre aujourd'hui, et dans quelles conditions?”. Et l'histoire apporte alors une donnée intéressante: le musée. L'idée de regrouper les \oe{}uvre, de les mettre dans un contexte non plus de décoration (ou de culte) mais dans un contexte d'exposition constitue en fait la réelle invention du concept artistique d'aujourd'hui. L'art n'est pourtant pas une invention récente, le terme vient du latin ars/artis dont le sens se rapproche plus de l’artisanat dans la conception contemporaine de l’art. Un terme ancien, donc, pour un concept qui n’est plus le même aujourd’hui. Il y a bien eu une évolution  
