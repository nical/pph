\subsection{Evolution du support artistique : exemple par l'histoire d'une couleur}
\xspace
	L'évolution de la peinture à travers les siècles est très liée à l'histoire des couleurs. Il est important d'observer l'évolution des couleurs dans les m\oe{}urs, les techniques et l'utilisation plastique. Le pigment est le support de l'expression artistique. C'est lui que l'artiste étale sur sa toile et c'est lui que nous, spectateurs, observons. Son agencement compose l'\oe{}uvre, il véhicule le geste et l'expression de l'artiste. Or, la création de pigments n'a pas toujours été aussi maîtrisée qu'aujourd'hui. À cela ajoutons que les couleurs ont eu, au cours de leur histoire, différents statuts sociaux, et même moraux. Tout cela entraîne d'importantes répercussions sur l'histoire de l'art. L'étude des couleurs représente donc un point de départ à la compréhension de l'évolution de la peinture. Etudions de façon non exhaustive l'histoire de la couleur bleu, la couleur dont le statut a le plus évolué au cours du temps.

\xspace
	Michel Pastoureau explique dans \emph{Bleu, histoire d’une couleur} qu’on date à trois mille ans avant notre ère les plus anciennes activités de teinture sur support textile en Europe parvenues jusqu'à nous. Toutes dans des tons rouges. La primauté du rouge est encore effective dans la Rome antique, alors que la couleur bleue, elle, est presque inexistante, à tel point que certains historiens ont soutenu que les grecs et les romains ne voyaient pas cette couleur. Il est en effet difficile de traduire en grec et en latin le bleu. En grec les deux mots les plus utilisés pour désigner le bleu sont \emph{glaukos} et \emph{kyaneos}; le dernier qualifie aussi bien le bleu clair des yeux que le noir des habits de deuil, mais jamais la couleur du ciel ou de la mer. Dans la période classique il désigne un couleur sombre: bleu foncé mais aussi violet, le noir le brun. \emph{Kyaneos} traduit d'avantage le sentiment de la couleur que sa teinte. \emph{Glaucos}, terme un peu plus ancien, désigne autant le gris que le vert, parfois même des teintes brunes. Il traduit d'avantage l'idée de pâleur qu'une réelle coloration. Dans l'\emph{Iliade} et l'\emph{Odyssée} d'Homère, on note que sur soixante adjectifs qualifiants des éléments de paysages, seulement trois se rapportent à la couleur quand de nombreux autres traitent de la lumière. En latin il existe de nombreux termes (glaucus,caeruleus, caesius, cyaneus, lividus, venetus, aerius, ferreus) mais ils sont tous imprécis, polysémiques et d'emplois contradictoires. Ce n'est qu'au Moyen Âge chrétien que deux mots apparaissent dans le lexique latin: blavus, venu des langues germaniques, et azureus, venu de l'arabe. Ce sont ces deux mots qui prendront le pas sur tous les autres dans les langues romanes. Certains savants ont donc mis en avant les théories évolutionistes pour appuyer que les grecs et, à leur suite, les romains ne voyaient pas la couleur bleue. Cependant cette thèse est, comme le souligne Michel Pastoureau, trop ethnocentriste et on oublie souvent qu'il y a un écart entre la vision qui est un phénomène biologique, et la perception qui est un phénomène en grande partie culturel. Les romains n'ignorent pas la couleur bleue, ils lui sont indifférents, voire hostiles : pour eux le bleu est la couleur des Barbares, Celtes et Germains.

\xspace
	Dans les images et \oe{}uvres d'art du haut Moyen Âge, le bleu est certes discret mais joue parfois des rôles importants. Bien que n'étant qu'une couleur périphérique sans symbolique propre pendant la période paléochrétienne, il devient dans l'empire Carolingien un moyen de mettre en valeur la majesté des souverains et même quelques fois une teinte des habits de personnages divins tels que l’empereur ou la Vierge. À partir du XIe siècles, le bleu change de statut. Il n'est plus désormais une couleur de second plan, mais devient une couleur à la mode, sa place en art se fait de plus en plus importante, “la plus belle couleur” selon certains auteurs. Cette promotion de la couleur bleue se ressent particulièrement dans son statut pictural et iconographique. Dans la palette de l'artiste, le bleu sombre, assez rare chez les cultures anciennes, s'éclaircit, se fait plus séduisant.

\sideimage{images/botticelliVe.jpg}{16em}{\emph{La Vierge et l’enfant entourés de cinq anges}, Botticelli, 1470}
\xspace
	Notons que la Vierge n'a pas toujours été représentée dans une robe bleue, il faut attendre la fin du XIIe siècle pour qu'elle soit principalement associée à cette couleur. Ce changement n'est pas anodin, Marie est l'une des figures les plus importantes dans un art alors presque exclusivement religieux. Pendant cette période le bleu change de statut mais aussi de teinte, les contraintes techniques et financières concernant la production des pigments sont encore importantes. On utilise de plus en plus souvent du cuivre ou du manganèse au lieu du cobalt. Le bleu gothique de la Sainte-Chapelle vers 1250 n'a pas grand rapport avec le bleu roman de Chartres du siècle précédent.

\xspace
	Pendant la révolte protestante, ont lieu de nombreuses réformes iconoclastes, mais aussi dans un certaine mesure “chromoclastes” : on s'insurge contre les couleurs exubérantes, trop voyantes au profits d'agencements sobres. Le bleu opère comme un retour dans l'histoire : il s'assombrit et se désature. Au XVIe siècle, c'est chez Calvin que l'on trouve le plus de recommandations à propos de l'art et de la couleur. Calvin ne condamne pas les arts plastiques mais il prône que ceux-ci doivent uniquement chercher à instruire et à honorer Dieu. Non pas en représentant le créateur, ce qui est abominable, mais en représentant la création. L'artiste doit donc fuir les sujets artificiels, gratuits ou invitant à l'intrigue. L'art n'a pas de valeur en soi : il vient de Dieu et doit aider à mieux le comprendre. Pour Calvin, les éléments constructifs de la beauté sont la clarté, l'ordre et la perfection. Chez certains peintres calvinistes on observe presque un phénomène de chromophobie. Rembrandt, par exemple, s'appuyait sur des tons foncés, peu nombreux (jusqu'à parfois tendre vers la monochromie), pour laisser place à de puissants effets de lumière et de vibration. Au XVIIe siècles l'austérité chromatique se retrouve aussi chez des peintres catholiques, en particulier ceux qui s'inscrivent dans le courant janséniste. On remarque par exemple que la palette de Philippe de Champaigne devient plus sobre lorsqu'il se rapproche du jansénisme. Sa palette se rapproche de celle de Rembrandt (avec le bleu en plus) et s'éloigne de celles de peintres faisant toujours usage de couleurs saturées, comme Rubens ou Van Dyck.

\xspace
	La diversité des palettes au XVIIe siècle reflète bien le débat qui oppose depuis la Renaissance l'importance de la couleur par rapport à celle du dessin. Les adversaires de la couleur soutiennent que le dessin est plus noble que la couleur car il est une création de l'esprit et non un simple mélange de matière. 
De plus la couleur est trompeuse, séduisante, elle détourne du vrai et du bien, et ne peut être qu'artifice et fausseté Par ailleurs elle est jugée incontrôlable dans le sens où  elle se refuse au langage et échappe à l'analyse. On ne comprend alors pas encore les phénomènes physiques liés à la couleur, mais les découvertes de Newton en optique montrent entre autres que la couleur est lumière, et qu'il est possible de la mesurer. Les artistes se rangent alors de l'avis des savants au profit de la couleur. La couleur permet de guider le regard, hiérarchiser les éléments, elle montre que le dessin seul ne parvient pas à tout représenter. 

\sideimage{images/titienHommeGants.jpg}{14em}{L’homme aux gants}{Titien, 1523}

	Prenons l'exemple du rendu de la chair dans lequel certains artistes tels que Titien ont excellé à l'aide de subtiles nuances de couleur. Par dessus tout, la couleur donne vie aux êtres de chair. Par la suite, la couleur bleue ne cesse de prendre de l'ampleur, on découvre de nombreux pigments de tons et d'intensités variés, le bleu s'impose en art, mais aussi dans les textiles et les textes. Aujourd’hui le bleu est la couleur préférée, loin devant toutes les autres. 
Yves Klein privilégie longtemps le bleu outremer dans ses peintures monochromes « Je suis allé signer mon nom au dos du ciel dans un fantastique voyage... » et il ira jusqu’à déposer, en 1960, la formule de ce bleu à l’Institut National de la Propriété Industrielle. L’importance du bleu est également visible au niveau vestimentaire. Le bleu est une couleur évoquant le futur, une couleur préférentielle dans le milieu de la mode, évoquant souvent la pureté du ciel et de l'eau.

\xspace
\centeredimage{images/kleinGlobe.jpg}{7cm}{Globe terrestre bleu}{Yves Klein, 1962}


