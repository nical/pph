\subsection{L'art comme moyen d'expression et de réflexion}
\xspace

	Nous avons vu que l'objet d'art était marqué par son contexte de création. Chaque \oe{}uvre s'inscrit dans un courant artistique qui se situe à une époque de l'Histoire de l'art. Mais l'\oe{}uvre d'art n'est pas le sujet de ce rapport au contexte, elle en est plutôt l'objet. Le sujet est en fait l'artiste, car c'est lui dont le processus créatif est guidé par les coutumes et les m\oe{}urs de son époque. Cette habitude, de confondre l'artiste et son travail est un signe évident du rapport étroit entre l'artiste et son \oe{}uvre. Chez les peintres du XVIIe, les diversités de palettes sont davantage dues à la façon de travailler et de mettre en \oe{}uvre les couleurs qu'à l'emploie de pigments différents. Elles sont aussi le reflet de sensibilités religieuses différentes: non seulement il y a une peinture catholique, protestante, mais dans chacune s'expriment des tendances ou de intensions plus nettement jésuites ou encore jansénistes.

\sideimage{images/schieleAutoportrait.jpg}{4.5cm}{Autoportrait}{Egon Schiele, 1910}

	La création plastique a une dimension personnelle intense. L'artiste se cherche, il s'étudie en même temps qu'il se perfectionne. Prenons l'exemple des très nombreux autoportraits que l'histoire de l'art nous présente. En se plaçant comme sujet, le peintre dirige son étude, son questionnement sur lui-même et l'immense diversité des autoportraits témoigne de la personnalité de chaque travail, de chaque questionnement de chaque façon d'appréhender le retour sur soi. Autoportraits de face, de dos (Lartigue), à l‘envers (Baselitz), nu (Egon Schiele ou Suzanne Valadon), doubles ou dans un miroir (Dubuffet), déguisé (Malévitch comme Van Dongen), avec un masque (Popovic), la diversité n'est pas uniquement portée par le choix de la pose ou la mise en situation. Baselitz, Buffet, César, Degas, Derain, Dubuffet, Max Ernst, Giacometti, Frida Kahlo, Fernand Léger, Malevitch, Matisse, Miró, Mondrian, Picabia, Picasso, Vasarely, Vlaminck, Vuillard, pour ne citer que des artistes exposés en 2004 au musée du Luxembourg à l'occasion de l'exposition "MOI ! - Autoportraits du XXème siècle –", présentent une extraordinaire diversité dans un domaine si précis, et ce au cours d'un seul siècle ! L'étude de ces autoportraits dépasserait largement le cadre de ce dossier, aussi contentons-nous de voir à quel point la peinture permet à l'artiste de s'explorer, de se faire sujet de sa réflexion. Cette réflexion ne s'arrête pas au niveau personnel. Par exemple, les vanités que nous évoquerons par la suite sont des réflexions sur la condition humaine, plus précisément sur la vanité de la vie face au statut de mortel.


%\sideimageleft{images/malevitchAutoportrait.jpg}{4cm}{Autoportrait}{Casimir Malévitch, 1933}
