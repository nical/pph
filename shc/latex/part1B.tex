\subsection{L'objet d'Art évolue dans le temps}	
\xspace
	Cet exemple de la couleur bleue nous montre plusieurs choses. Premièrement, l'art a évolué, les \oe{}uvres produites à l'époque d'Homère son nettement différentes des oeuvres produites au XIIe siècles. L'histoire du bleu nous montre que le rapport entre l'homme et l'art du point de vue du processus créatif est très lié au contexte de l’époque, que l'art évolue. À l'époque d'Homère il n'y a pas de bleu en Art, et ce même en poésie, ou en chants; alors qu'au XIIe siècle le bleu est une couleur importante, couleur des rois et de la Vierge. Cette aspect certes très formel reflète les coutumes et l'arrière plan psychologique de l'artiste. Ainsi les m\oe{}urs changent, évoluent, et il en va donc de même pour l'objet d'art.

 	De quoi se compose un tableau? D'une somme de courbes, d'aplats : la matière picturale “agencée d'une certaine façon”. Cet agencement traduit le geste de son auteur, l'intention du peintre mais aussi un arrière plan psychologique : la vie morale, les sensations que le peintre a associé à sa création. Tout cela fait du tableau un objet complexe; complexe en soi, du point de vue de sa création, mais complexe également dans le regard qui est porté sur lui. C'est un point important car c'est celui qui transparaît le plus aujourd'hui : l'interprétation, la mise dans un contexte nouveau, l'observation à travers une culture qui a évolué. Une \oe{}uvre d'art ancienne n'est pas exactement une création artistique mais plutôt le reflet d'une création artistique. Nous contemplons un tableau de Rubens déformé par le temps, comme si une vitre s'interposait et déformait non pas l'image physique mais l'image psychologique et sensible que renvoient l'\oe{}uvre. Cette barrière nous empêche d'interroger le créateur alors on étudie les traces qu'il a laissées. Les incertitudes augmentent et les interprétations divergent. Chaque historien d'art (ces derniers étant les autorités dans le domaine) peut interpréter l'\oe{}uvre en fonction de sa sensibilité. À ce moment là comment se positionner ? L'\oe{}uvre d'art a-t-elle autant de “sens” que d'interprétations ? A-t-elle seulement un “sens” particulier ? Le contact entre l'\oe{}uvre et l'homme est d'autant plus subjectif que l'homme ne sait avec certitude que peu de chose à propos de l'\oe{}uvre qu'il contemple. Ce mystère qui plane sur l'objet d'art ancien constitue une dimension nouvelle. Une dimension portée par le temps. Face à cela on peut prendre l'\oe{}uvre pour référentiel et considérer que le spectateur a évolué ainsi que son contact avec l'\oe{}uvre. On peut tout autant prendre l'homme comme référentiel et considérer que c'est l'objet artistique qui a évolué. Dans un cas comme dans l'autre, l'homme aujourd’hui n'a pas le même rapport avec un objet d'art donné qu'un contemporain de l'auteur. Ce rapport au temps tient donc une place très importante. 
