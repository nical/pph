	Le rapport entre l’artiste et son œuvre est donc guidé par le processus psychologique sous-jacent à la création : la recherche de transcendance. La peur de la mort pousse l’homme à trouver des moyens de dépasser sa condition corporelle et mortelle : l’homme, et en particulier l’artiste, repousse les limites de son corps et invente l’Histoire. De plus toute l’abstraction mise en œuvre dans le processus artistique entraîne une forme de réflexion, souvent transmis intentionnellement par l’artiste. Une autre forme de dépassement se situe dans la capacité à transmettre par le biais d’un moyen original : l’œuvre d’art, qui se munit d’un langage visuel comme nous allons le voir par la suite. L’art est donc aussi un moyen d’expression et de réflexion, l’art possède une dimension médiatique. Ainsi l'artiste crée-t-il pour lui même ou crée-t-il pour ceux qui regarderont son œuvre ?  

