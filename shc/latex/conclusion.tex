	Le dialogue  entre l’homme et l’art est riche et complexe. Les fruits de ce dialogue sont l’émotion, l’interpellation, le questionnement. L’œuvre d’art fait appel à notre capacité d’abstraction mais aussi à notre psychologie. Elle est là pour nous interpeller ainsi que pour transmettre, qu’il s’agisse du message de l’artiste ou d’une vision sociale à un moment donné. L’œuvre d’art nous incite à observer un ailleurs, à déchiffrer des signes. C’est un média utilisé par l’artiste et ses commanditaires : église, puissance publique pour transmettre des visions, des messages.
L’œuvre d’art nous provoque et elle nous construit. Nous la regardons et nous la déformons, voyant en elle ce que nous voulons ou pouvons y voir dans notre propre contexte et avec le regard de notre époque.  En ce sens le dialogue entre l’homme et l’art évolue dans le temps. L’œuvre d’art nous questionne  sur cet ailleurs qu’elle représente et sur cet autre qu’elle portraitise. Elle nous interroge également sur cet autre qu’est  l’artiste, qui l’a produit. 
L’artiste met toute son énergie pour susciter l’émotion. Pour créer, il se dépasse et  cette transcendance devient sa raison d’être. L’artiste crée pour les autres, pour exprimer, protester, transmettre, donner du plaisir,  mais il crée aussi pour lui même : dépasser la frontière de la mort en laissant une œuvre et un nom. 
Ce qui a été voulu par l’artiste n’est pas toujours ce que nous percevons et à un moment donné. Des années, des siècles plus tard ou des kilomètres plus loin, ce qui relie l’artiste au à l’homme qui regarde son œuvre c’est l‘émotion. L’émotion est l’objet central et permanent de l’art et tout ce qui est mis en œuvre dans la création artistique est investi pour susciter l’émotion ; émotion intellectuelle ou sensuelle. C’est par elle que nous allons « rentrer dans l’œuvre » et cheminer. 
